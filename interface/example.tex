\documentclass{standalone}
\usepackage[utf8]{inputenc}
\usepackage{amsmath,amssymb}

%% logical symbols in the amsmath style: already have \land, \lor, \lnot
\newcommand{\lif}{\rightarrow}      % implication
\newcommand{\liff}{\leftrightarrow} % bi-implication
\newcommand{\lfalse}{\bot}          % contradiction
\newcommand{\lall}{\forall}         % universal quantification
\newcommand{\lis}{\exists}          % existential quantification

%%%% COPY THIS PART TO YOUR LaTeX PREAMBLE %%%%
%%%% Provide a \fitch{...}{...} command for formal proofs
\setbox0=\hbox{\begin{tabular}{r@{}|}\end{tabular}}
\newlength\fitchone    \fitchone 6cm
\newlength\fitchtwo    \fitchtwo 0pt
\newlength\fitchthree  \fitchthree 0pt
\advance\fitchthree by 10pt
\newcommand{\fitch}[2]{\advance \fitchone by -\fitchthree%
\hspace*{.35em}\begin{tabular}[t]{|p{\fitchtwo}@{}p{\fitchone}@{}l}
  \multicolumn{3}{@{}l@{}}{\ }\\[-2.35ex]
   #1 \\
   \ \\[-2.5ex] \cline{1-1}\\[-2ex]
   #2 \\ \multicolumn{3}{@{}l@{}} \ \\[-2.35ex]
  \end{tabular}}

\begin{document}

% The \fitch command takes two arguments: \fitch{...}{...}
% The first argument contains the premises and the second the rest of the derivation.
% All lines start with the symbol &.

% If any part has more than one line you should use \\ to separate them.

% To keep your LaTeX code readable, indent all lines after \fitch{
% and further indent subproofs.

\fitch{
  & 1. $P \lif Q$ \\
  & 2. $Q \lif R$ % watch out: do not write "& \\" here
  }{ % here is the first short separating line
  \fitch{ % here we start a subproof, it helps to also indent it
    & 3. $P$
    }{ % second short separating line
    & 4. $Q$       & $\lif$ Elim: 1, 3 \\
    & 5. $R$       & $\lif$ Elim: 2, 4
    } \\
  & 6. $P \lif R$  & $\lif$ Intro: 3--5
}

\end{document}
